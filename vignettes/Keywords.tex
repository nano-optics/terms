% Options for packages loaded elsewhere
\PassOptionsToPackage{unicode}{hyperref}
\PassOptionsToPackage{hyphens}{url}
%
\documentclass[
]{article}
\usepackage{amsmath,amssymb}
\usepackage{lmodern}
\usepackage{ifxetex,ifluatex}
\ifnum 0\ifxetex 1\fi\ifluatex 1\fi=0 % if pdftex
  \usepackage[T1]{fontenc}
  \usepackage[utf8]{inputenc}
  \usepackage{textcomp} % provide euro and other symbols
\else % if luatex or xetex
  \usepackage{unicode-math}
  \defaultfontfeatures{Scale=MatchLowercase}
  \defaultfontfeatures[\rmfamily]{Ligatures=TeX,Scale=1}
\fi
% Use upquote if available, for straight quotes in verbatim environments
\IfFileExists{upquote.sty}{\usepackage{upquote}}{}
\IfFileExists{microtype.sty}{% use microtype if available
  \usepackage[]{microtype}
  \UseMicrotypeSet[protrusion]{basicmath} % disable protrusion for tt fonts
}{}
\makeatletter
\@ifundefined{KOMAClassName}{% if non-KOMA class
  \IfFileExists{parskip.sty}{%
    \usepackage{parskip}
  }{% else
    \setlength{\parindent}{0pt}
    \setlength{\parskip}{6pt plus 2pt minus 1pt}}
}{% if KOMA class
  \KOMAoptions{parskip=half}}
\makeatother
\usepackage{xcolor}
\IfFileExists{xurl.sty}{\usepackage{xurl}}{} % add URL line breaks if available
\IfFileExists{bookmark.sty}{\usepackage{bookmark}}{\usepackage{hyperref}}
\hypersetup{
  pdftitle={keywords},
  hidelinks,
  pdfcreator={LaTeX via pandoc}}
\urlstyle{same} % disable monospaced font for URLs
\usepackage[margin=1in]{geometry}
\usepackage{graphicx}
\makeatletter
\def\maxwidth{\ifdim\Gin@nat@width>\linewidth\linewidth\else\Gin@nat@width\fi}
\def\maxheight{\ifdim\Gin@nat@height>\textheight\textheight\else\Gin@nat@height\fi}
\makeatother
% Scale images if necessary, so that they will not overflow the page
% margins by default, and it is still possible to overwrite the defaults
% using explicit options in \includegraphics[width, height, ...]{}
\setkeys{Gin}{width=\maxwidth,height=\maxheight,keepaspectratio}
% Set default figure placement to htbp
\makeatletter
\def\fps@figure{htbp}
\makeatother
\setlength{\emergencystretch}{3em} % prevent overfull lines
\providecommand{\tightlist}{%
  \setlength{\itemsep}{0pt}\setlength{\parskip}{0pt}}
\setcounter{secnumdepth}{-\maxdimen} % remove section numbering
\usepackage{mathrsfs}
\ifluatex
  \usepackage{selnolig}  % disable illegal ligatures
\fi

\title{keywords}
\author{}
\date{\vspace{-2.5em}}

\begin{document}
\maketitle

\newcommand{\bE}{\mathbf{E}}
\newcommand{\bB}{\mathbf{B}}
\newcommand{\ldoc}{\mathscr{C}}
\newcommand{\ldocbar}{\overline{\mathscr{C}}}
\newcommand{\ldocoa}{\langle\mathscr{C}\rangle}
\newcommand{\ldocoabar}{\langle\overline{\mathscr{C}}\rangle}

List of (case sensitive!) keywords and corresponding arguments supported
by \textbf{TERMS}. Optional arguments are enclosed in square brackets
(nested in some cases).

\hypertarget{main-input-parameters}{%
\subsection{Main input parameters}\label{main-input-parameters}}

\hypertarget{modeandscheme-m-s}{%
\subsubsection{\texorpdfstring{ModeAndScheme \emph{M}
\emph{S}}{ModeAndScheme M S}}\label{modeandscheme-m-s}}

If present, this keyword must appear first in the input file. It takes
two arguments: positive integer \emph{M} specifying the desired
calculation mode; and non-negative integer \emph{S} specifying the
solution scheme to be used. The default values are \emph{M} = 2 and
\emph{S} = 3.

\textbf{Mode of calculation}

\begin{itemize}
\tightlist
\item
  \emph{M = 1} triggers a single- or multi-wavelength calculation of
  near fields \(\mathbf{E}\), \(\mathbf{B}\) and optical chirality
  \(\mathscr{C}\), at fixed incidence directions and/or
  orientation-averaged
\item
  \emph{M = 2} triggers a single- or multi-wavelength calculation of
  far-field properties (e.g.~spectra of optical cross-sections), at
  fixed incidence directions and/or orientation-averaged
\item
  \emph{M = 3} triggers a single- or multi-wavelength calculation of
  polarimetric properties, such as Stokes scattering vectors, phase
  matrix, and differential scattering cross sections for multiple
  incidence and/or scattering angles
\end{itemize}

\textbf{Scheme of solution}

\begin{itemize}
\tightlist
\item
  \emph{S = 0} will seek solutions for the given angles of incidence,
  without seeking the collective \emph{T}-matrix
\item
  \emph{S \textgreater{} 0} will calculate the collective
  \emph{T}-matrix either (\emph{S = 1}) by direct inversion of the
  complete linear system to obtain \(T^{(i,j)}\), or (\emph{S = 2}) by
  using Stout's iterative scheme for \(T^{(i,j)}\), or (\emph{S = 3}) by
  using Mackowski's scheme for \(T^{(i)}\). Note that fixed-orientation
  cross-sections are also calculated when \(S > 0\).
\end{itemize}

\hypertarget{scatterers-n}{%
\subsubsection{\texorpdfstring{Scatterers
\emph{N}}{Scatterers N}}\label{scatterers-n}}

This keyword must appear last in the \texttt{inputfile}, with a single
positive integer argument \emph{N} specifying the number of scatterers.
The following \emph{N} lines specify all the required parameters per
scatterer, and each line must contain five or more space-separated
fields, i.e.

\begin{verbatim}
Tag x y z R [ a b c [ d ] ]     ( if Tag(1:2)  = "TF" )   
            [ a [ b [ c ] ] ]   ( if Tag(1:2) != "TF" )
\end{verbatim}

where \emph{Tag} is a contiguous string, which may contain one
underscore to separate the root from the suffix; \emph{x}, \emph{y},
\emph{z} are the Cartesian coordinates (in the lab frame) for the
scatterer, whose smallest circumscribing sphere has radius \emph{R}. All
other subsequent parameters (inside square brackets) depend on the root
of \emph{Tag} .

Before the root of \emph{Tag} is parsed, the code first looks for a
suffix of the form \emph{\_S?} and associates it with a multipole
selection predefined using the
\protect\hyperlink{MultipoleSelections}{\texttt{MultipoleSelections}}
keyword.

If the root of \emph{Tag} is either ``TF1'', ``TF2'', \ldots{} , or
``TF9'', which correspond to a 1-body \emph{T}-matrix file listed under
the \protect\hyperlink{TmatrixFiles}{\texttt{TmatrixFiles}} keyword,
floats \emph{a}, \emph{b}, and \emph{c} can be supplied to specify the
Euler angles describing the scatterer orientation (default angle values
are all zero). Another float \emph{d} may be included to specify the
aspect ratio for spheroids, which is currently only used for mapping
local field intensity and visualising the geometry. Note that \emph{d}
is interpreted as the ratio of polar (along z) to equatorial (along x or
y) length, so that \emph{d \textgreater{} 1} for prolate spheroids,
\emph{d \textless{} 1} for oblate spheroids, and \emph{d = 1} for
spheres (default). Note that for nonspherical particles the
circumscribing sphere radius \emph{R} is used to check for potential
geometrical overlap between particles, but also in the balancing scheme.

If the root of \emph{Tag} is not ``TF?'', the 1-body \emph{T}-matrix is
computed using Mie theory, which is applicable to coated spheres. The
expected \emph{Tag} format is \texttt{L0@L1@L2@L3}, with the character
``@'' delimiting substrings that specify the material dielectric
function of each concentric region inside the scatterer, starting from
the core (\emph{L0}) and going \emph{outward}. The number of coats is
inferred from the number of instances of ``@'' and is currently capped
at 3. \emph{Tag} of a homogeneous sphere (without layers) should not
contain any ``@'', i.e.~\emph{Tag = L0}. Currently accepted values for
\emph{L?} are: ``Au'', ``Ag'', ``Al'', ``Cr'', ``Pt'', ``Pd'', ``Si'',
and ``Water'' which trigger internal calculation of the
wavelength-dependent dielectric functions for the required material, or
``DF1'', ``DF2'', \ldots, ``DF9'' to impose a custom dielectric function
listed under the
\protect\hyperlink{dielectricfunctions-nfuns-1}{\texttt{DielectricFunctions}}
keyword. For coated spheres, the outer radius of each region must be
specified by floats \emph{R}, \emph{a}, \emph{b}, \emph{c} in the order
of decreasing size (i.e.~going radially \emph{inward}).

\hypertarget{tmatrixfiles-nfiles}{%
\subsubsection{\texorpdfstring{TmatrixFiles
\emph{Nfiles}}{TmatrixFiles Nfiles}}\label{tmatrixfiles-nfiles}}

Specifies the number of external \emph{T}-matrix files (default:
\emph{Nfiles = 0}). The subsequent \emph{Nfiles} lines are each read as
a string and then interpreted as a filename. Wrap the string in
quotation marks if it contains the relative path or special characters,
e.g.~\texttt{"../../tmatrix\_Au\_spheroid\_50x20\_water.tmat"}. Note
that the wavelengths in each file must \emph{exactly} correspond to the
values specified by the
\protect\hyperlink{wavelength-l1-l2-n-}{\texttt{Wavelength}} keyword.

The \emph{T}-matrix file format is as follows:

\begin{itemize}
\tightlist
\item
  First line is a comment (starts with a \texttt{\#}) describing the
  format \texttt{\#\ s\ sp\ n\ np\ m\ mp\ Tr\ Ti}
\item
  Second line is also a comment and starts with
  \texttt{\#\ lambda=\ N1\ nelements=\ N2} where N1 is the wavelength in
  nanometres, and N2 is the number of \emph{T}-matrix elements to be
  read below
\item
  Subsequent lines contain the indices and \emph{T}-matrix values for
  this particular wavelength,
\end{itemize}

\begin{enumerate}
\def\labelenumi{\arabic{enumi}.}
\tightlist
\item
  \texttt{s}, \texttt{sp} are the row (resp. column) index of the
  multipole mode (1: magnetic, or 2: electric)
\item
  \texttt{n}, \texttt{np} index the multipole degree
\item
  \texttt{m}, \texttt{mp} index the multipole order
\item
  \texttt{Tr}, \texttt{Ti} give the real and imaginary part of the
  \emph{T}-matrix element
\end{enumerate}

\begin{itemize}
\tightlist
\item
  If the file contains multiple wavelengths each wavelength-block is
  appended below the others, starting with a line
  \texttt{\#\ lambda=\ N1\ nelements=\ N2}
\end{itemize}

An example is show below,

\begin{verbatim}
# s sp n np m mp Tr Ti | prolate Au spheroid in water, a = 10 c = 20
# lambda= 400 nelements= 136 epsIn= -1.649657+5.771763j
  1   1   1   1  -1  -1 -1.189109253832815e-04 -2.161746691903687e-05
  1   1   1   1   0   0 -5.597968829951113e-05 -3.444956295771378e-05
... [truncated]
  2   2   4   4   3   3 -3.794740062663782e-11 5.636725538124517e-11
  2   2   4   4   4   4 -1.113090425618089e-11 1.707927691863483e-11
# lambda= 402 nelements= 136 epsIn= -1.661947+5.778032j
  1   1   1   1  -1  -1 -1.160926707256971e-04 -2.119092055798298e-05
  1   1   1   1   0   0 -5.467319805259745e-05 -3.371696756234449e-05
... [truncated]
  2   2   4   4   3   3 -1.279170882307354e-15 1.378894188143029e-13
  2   2   4   4   4   4 -3.752182192799965e-16 4.101975575297762e-14
... [truncated]
# lambda= 800 nelements= 136 epsIn= -24.236565+1.458652j
  1   1   1   1  -1  -1 -7.146139984375531e-07 -1.120611667309835e-05
  1   1   1   1   0   0 -4.379156367712547e-07 -7.955074171282911e-06
... [truncated]
  2   2   4   4   3   3 -1.240958755455683e-15 1.346747233206165e-13
  2   2   4   4   4   4 -3.640885008022631e-16 4.006450678480949e-14
... [truncated]
\end{verbatim}

\hypertarget{dielectricfunctions-nfuns}{%
\subsubsection{\texorpdfstring{DielectricFunctions
\emph{Nfuns}}{DielectricFunctions Nfuns}}\label{dielectricfunctions-nfuns}}

Specifies the number of custom dielectric functions (default:
\emph{Nfuns = 0}). The subsequent \emph{Nfuns} lines are each read as a
string and then interpreted as either (i) a filename with a relative
path or (ii) real and imaginary parts of a constant (i.e.~wavelength
independent) value. Wrap each string in quotation marks,
e.g.~\texttt{"../../epsAg.dat"} or \texttt{"2.25d0\ 0.0d0"}. The files
should be in three-column format: the wavelength in nm followed by the
real and imaginary parts of the relative dielectric function on each
line. The wavelength range in the file must fully contain the range
specified by the
\protect\hyperlink{wavelength-l1-l2-n-}{\texttt{Wavelength}} keyword,
but the values need not correspond exactly as they will be linearly
interpolated.

\hypertarget{medium-x}{%
\subsubsection{\texorpdfstring{Medium
\emph{X}}{Medium X}}\label{medium-x}}

Sets the real-valued dielectric constant of the host medium (default
value is \emph{1.0}). If \emph{X \textless{} 0} then its magnitude is
interpreted as a refractive index (\emph{s}), from which the dielectric
constant is calculated as \(X=s^2\).

\hypertarget{wavelength-l1-l2-n}{%
\subsubsection{\texorpdfstring{Wavelength \emph{L}1 {[} \emph{L2}
\emph{n} {]}}{Wavelength L1 {[} L2 n {]}}}\label{wavelength-l1-l2-n}}

Without the optional arguments, this keyword changes the default
wavelength of 666.0 nm to a new value \emph{L}1. Including the optional
arguments will specify a closed interval {[} \emph{L}1, \emph{L}2 {]}
divided into \emph{n} regular grid spacings, thus producing \emph{n+1}
wavelengths.

\hypertarget{incidence-a-b-c-p-na-nb-nc}{%
\subsubsection{\texorpdfstring{Incidence \emph{a} \emph{b} \emph{c} {[}
\emph{p} {]} / {[} \emph{na} \emph{nb} \emph{nc}
{]}}{Incidence a b c {[} p {]} / {[} na nb nc {]}}}\label{incidence-a-b-c-p-na-nb-nc}}

or

\texttt{Incidence\ file\ filename\ {[}p{]}}

This keyword modifies the incident plane-wave. The default travel
direction (along \emph{z} in lab-frame) can be changed by the Euler
angles \emph{a} in the range \([0,2\pi)\) and \emph{b} in the range
\([0,\pi]\), coinciding with the azimuthal and the polar angles,
respectively, of the spherical polar coordinates in the lab frame. In
addition, the amplitude vector can then be rotated about the new travel
direction by the third Euler angle \emph{c} in the range \([0,2\pi)\).
All three Euler angles are defined in accordance with the right-hand
rule, and the sequence of rotation angles \emph{a},\emph{b},\emph{c}
corresponds to the intrinsic ZY'Z' convention. That is: rotate by
\emph{a} about the current \emph{z}-axis, then by \emph{b} about the new
\emph{y}-axis, and finally by \emph{c} about the new \emph{z}-axis.

Near-field and polarimetric calculations, i.e.~in modes \emph{M = 1} and
\emph{M = 3}, require the polarisation of incident light to be
specified. The polarisation is set by integer \emph{p}, with
\emph{\textbar p\textbar{} = 1} setting linear polarisation,
\emph{\textbar p\textbar{} = 2} setting circular polarisation, and the
sign selecting one of the two Jones vectors in each case (positive:
\emph{x}-linear-polarised or \emph{R}-circular-polarised; negative:
\emph{y}-linear polarised or \emph{L}-circular-polarised). Note: for a
circularly polarised wave travelling along \emph{z}, right-circular
(\emph{R}) polarisation means that the amplitude vector is rotating
clockwise in the \emph{xy}-plane from the receiver's viewpoint (looking
in the negative \emph{z} direction).

The integer \emph{p} can be omitted in mode \emph{M = 2}, because its
output is always calculated for all four polarisations.

A negative value of argument \emph{a}, \emph{b}, and/or \emph{c} will
trigger discretisation of the corresponding angle range to produce
\(-a\) grid points (resp. \(-b\) or \(-c\)). The grid points are
uniformly spaced for the first and the third Euler angles, but for the
second (i.e.~polar) angle the discretisation is such that the cosine is
uniformly spaced. The range maximum of each angle can be divided by an
(optional) integer \emph{na}, \emph{nb}, and \emph{nc}, to help avoid
evaluating redundant grid points in the presence of symmetry. Note that
the discretisation is constructed so that orientational averages are
computed as a uniformly weighted Riemann sum with the midpoint rule. The
weight \emph{wi} of each grid-point \emph{i} is simply
\(w_i = 1/n_\text{gps}\), where \emph{ngps} is the total number of grid
points.

Multiple incidences can also be read from a file, in which case the
argument \emph{a} must be a string starting with `f' or `F', and
\emph{b} must specify the filename. The file's first line must contain
the total incidence count, \emph{ninc}, and the subsequent \emph{ninc}
lines each must contain four space-separated values: the three Euler
angles (\emph{ai}, \emph{bi}, \emph{ci}) and the weight \emph{wi} of
each incidence. The weights are only used to compute rotational averages
for convenience, which is a common use-case.

In \texttt{Mode\ =\ 1} (near-field calculations), if \texttt{p} is set
to \texttt{p=1} (default value, linear polarisaiton), the orientation
average of the local degree of optical chirality
\(\langle\mathscr{C}\rangle\) will be calculated for both RCP and LCP
(noting that linear polarisation would give 0 everywhere, when
orientation-averaged). Since the calculation can be time-consuming,
setting \texttt{p=+/-2} triggers the calculation for only that specific
circular polarisation.

\hypertarget{multipolecutoff-n1-n2-t}{%
\subsubsection{\texorpdfstring{MultipoleCutoff \emph{n1} {[} \emph{n2}
{[} \emph{t} {]}
{]}}{MultipoleCutoff n1 {[} n2 {[} t {]} {]}}}\label{multipolecutoff-n1-n2-t}}

Change the primary multipole cutoff (used for irregular offsetting when
staging the linear system) from the default value of 8 to \emph{n1}.
Another cutoff (used for regular offsetting when ``contracting'' the
collective \emph{T}-matrix) can be set to \emph{n2} \textgreater=
\emph{n1} (equality by default). A relative tolerance \(10^t\) (with
\(t<0\) and \(t = -8\) by default) is used in the test for convergence
of cross-sections with respect to multipole order \(n=1\dots n_2\) (the
summation can terminate below \(n_2\) if the relative tolerance is
reached).

\hypertarget{multipoleselections-ns}{%
\subsubsection{\texorpdfstring{MultipoleSelections
\emph{Ns}}{MultipoleSelections Ns}}\label{multipoleselections-ns}}

This keyword defines optional multipole selections for individual
\emph{T}-matrices, and it must be followed by \emph{Ns} lines with two
fields: (i) a string \emph{range} specifying the selection range; and
(ii) a string \emph{type} specifying the selection type. For example:

\begin{verbatim}
MultipoleSelections 3
MM1:4_EM1:4_ME1:4_EE1:4  blocks
MM1:0_EM1:15_ME1:8_EE1:0  rows
EM1:1_ME1:1  columns
\end{verbatim}

The \emph{range} string must be of the form MM?:?\_EM?:?\_ME?:?\_EE?:?,
with the underscores separating the ranges for each \emph{T}-matrix
block (e.g.~MM or ME), and each range specified by a closed multipole
interval ?:? (e.g.~\emph{n}lo:\emph{n}hi = 1:4). No selection will be
applied to blocks not included in \emph{range}, so these ``missing''
blocks will remain unmasked (left ``as is'' in the original
\emph{T}-matrix). On the other hand, a whole block can be masked (zeroed
out) by setting \emph{n}lo \textgreater{} \emph{n}hi
(e.g.~\texttt{MM1:0} will set the whole \texttt{MM} block of the
\emph{T}-matrix to 0).

The \emph{type} string must either start from ``c'', ``r'', or ``b'', to
indicated that the selection is either applied to \emph{T}-matrix
columns, rows, or both (producing non-zero blocks). To clarify, if
\emph{type(1:1) = ``c''}, then all \emph{T}-matrix columns corresponding
to multipole orders \emph{n \textless{} n}lo and \emph{n \textgreater{}
n}hi will be set to zero. For \emph{type(1:1) = ``b''}, columns
\textbf{and} rows for \emph{n \textless{} n}lo and \emph{n
\textgreater{} n}hi will be set to zero.

\hypertarget{output-control}{%
\subsection{Output control}\label{output-control}}

\hypertarget{outputformat-f-filename}{%
\subsubsection{\texorpdfstring{OutputFormat \emph{F} {[} \emph{filename}
{]}}{OutputFormat F {[} filename {]}}}\label{outputformat-f-filename}}

If present, the output file format \emph{F} can be switched between
plain text (``TXT'', default) and HDF5 (``HDF5''). With ``HDF5'', the
results will be stored in a file with name ``results.h5'', or a
user-specified filename (extension \emph{.h5} added automatically).

\hypertarget{verbosity-l}{%
\subsubsection{\texorpdfstring{Verbosity
\emph{L}}{Verbosity L}}\label{verbosity-l}}

Keyword specifying integer-valued verbosity level \emph{L}. Silent mode
(\emph{L = 0}) prints only error statements and warnings. Physical
quantities and some status indicators are printed at low verbosity
(\emph{L = 1}, default value), with various timings and convergence
indicators released at medium verbosity (\emph{L = 2}). The highest
level (\emph{L = 3}) is intended for debugging, releasing all print
statements throughout the code.

\hypertarget{near-field-specific-keywords}{%
\subsection{Near-field specific
keywords}\label{near-field-specific-keywords}}

\hypertarget{spacepoints-filename}{%
\subsubsection{\texorpdfstring{SpacePoints
\emph{filename}}{SpacePoints filename}}\label{spacepoints-filename}}

or

\texttt{SpacePoints} \emph{xlo} \emph{xhi} \emph{nx} \emph{ylo}
\emph{yhi} \emph{ny} \emph{zlo} \emph{zhi} \emph{nz}

Read (from a file) or calculate (on a regular grid) the Cartesian
coordinates of points in space, where the local field quantities are to
be evaluated. The file's first line should contain the total number of
space-points, and the subsequent lines must contain the \emph{x},
\emph{y}, and \emph{z} coordinates of each point. A regular grid is
specified by a closed interval, e.g.~\emph{{[} xlo, xhi {]}}, and the
number of bins (\emph{nx}) the interval is to be divided into (thus
producing \emph{nx+1} grid points along that dimension).

\hypertarget{mapquantity-p-e-b-c}{%
\subsubsection{\texorpdfstring{MapQuantity {[}\emph{p}{]} {[}\emph{E}{]}
{[}\emph{B}{]}
{[}\emph{C}{]}}{MapQuantity {[}p{]} {[}E{]} {[}B{]} {[}C{]}}}\label{mapquantity-p-e-b-c}}

Specify the near-field quantities of interest, in \texttt{Mode\ =\ 1}.
Integer argument \emph{p} selects the raising power applied to the field
amplitude \(|\mathbf{E}|^p\) or \(|\mathbf{B}|^p\). The default is
\emph{p = 2} to produce the field intensity, \emph{p = 1} is for the
field amplitude \textbar E\textbar, \emph{p = 4} for the (approximate)
Raman enhancement factor \emph{\textasciitilde\textbar E\textbar{}4}.
Setting \emph{p = 0} will output the real and imaginary parts of the
(vector!) field components at each space-point.

The optional letters {[}\emph{E}{]} {[}\emph{B}{]} {[}\emph{C}{]}
(default: \emph{E} only) determine which of the near-field properties
(electric and magnetic fields and normalised value of local degree of
optical chirality) will be calculated.

\hypertarget{mapoaquantity-e-b-c}{%
\subsubsection{\texorpdfstring{MapOaQuantity {[}\emph{E}{]}
{[}\emph{B}{]}
{[}\emph{C}{]}}{MapOaQuantity {[}E{]} {[}B{]} {[}C{]}}}\label{mapoaquantity-e-b-c}}

This is a keyword applicable in \texttt{Mode\ =\ 1}, to request the
calculation of analytical orientation-averaged near-field quantities
\(\langle|\mathbf{E}|^2\rangle\), \(\langle|\mathbf{B}|^2\rangle\), or
\(\langle\overline{\mathscr{C}}\rangle\). If this keyword is not
included in the input file, by default none will be calculated. Note
that the
\protect\hyperlink{incidence-a-b-c-p-na-nb-nc-}{\texttt{Incidence}}
keyword is used to select LCP and RCP (or both).

\hypertarget{polarimetry-keywords}{%
\subsection{Polarimetry keywords}\label{polarimetry-keywords}}

\hypertarget{scatteringangles-a-b-c-na-nb-nc}{%
\subsubsection{\texorpdfstring{ScatteringAngles \emph{a} \emph{b}
\emph{c} / {[} \emph{na} \emph{nb} \emph{nc}
{]}}{ScatteringAngles a b c / {[} na nb nc {]}}}\label{scatteringangles-a-b-c-na-nb-nc}}

This keyword specifies the scattering angles in \texttt{Mode\ =\ 3}
(polarimetry), for the calculation of Stokes scattering vectors at
different scattering angles. The parameters have the same interpretation
as for
\protect\hyperlink{incidence-a-b-c-p-na-nb-nc-}{\texttt{Incidence}}.

Multiple scattering angles can also be read from a file, in which case
the argument \texttt{a} must be a string starting with `f' or `F', and
\texttt{b} must specify the filename. The file's first line must contain
the number of scattering angles, \emph{nsca}, and the subsequent
\emph{nsca} lines each must contain three space-separated values: the
three Euler angles (\emph{ai}, \emph{bi}, \emph{ci}) for each scattering
angle.

\begin{center}\rule{0.5\linewidth}{0.5pt}\end{center}

\hypertarget{advanced-use-development}{%
\subsection{Advanced use / development}\label{advanced-use-development}}

\hypertarget{scatterercentredcrosssections}{%
\subsubsection{ScattererCentredCrossSections}\label{scatterercentredcrosssections}}

Applicable in Scheme 1 and 2.

Triggers Stout's formulae for fixed and orientation-averaged
cross-sections based on scatterer-centred matrices; otherwise, the
default behaviour is to collapse the coefficients to a common origin.
Note that this does not affect the calculation of fixed-orientation
partial shell absorptions for layered spheres, as they are calculated
separately.

\hypertarget{dumpcollectivetmatrix-filename}{%
\subsubsection{\texorpdfstring{DumpCollectiveTmatrix {[} \emph{filename}
{]}}{DumpCollectiveTmatrix {[} filename {]}}}\label{dumpcollectivetmatrix-filename}}

If the collective \emph{T}-matrix is computed, this keyword will dump it
to a file ``tmat\_col.txt'' or a user-specified \emph{filename}. The
\protect\hyperlink{tmatrixfiles-nfiles-1}{file format} is
self-consistent, so that the generated \emph{T}-matrix can be fed back
into \textbf{TERMS} for subsequent calculations.

\hypertarget{dumpprestageda}{%
\subsubsection{DumpPrestagedA}\label{dumpprestageda}}

If present, dumps a sparse-format representation of the full matrix
comprising the individual \emph{T}-matrices after potential masking
followed by rotation in their respective frame.

\hypertarget{dumpstageda}{%
\subsubsection{DumpStagedA}\label{dumpstageda}}

If present, dumps a sparse-format representation of the full matrix
comprising the individual \emph{T}-matrices along the diagonal blocks,
and translation matrices in the off-diagonal blocks. The exact form of
this matrix is scheme-dependent.

\hypertarget{dumpscacoeff}{%
\subsubsection{DumpScaCoeff}\label{dumpscacoeff}}

If present, dumps the scattering coefficients in to a file
``Sca\_coeff'' for different incidence angles.

\hypertarget{dumpinccoeff}{%
\subsubsection{DumpIncCoeff}\label{dumpinccoeff}}

If present, dumps the incident coefficients in to a file ``Inc\_coeff''
for different incidencd angles.

\hypertarget{disablestoutbalancing}{%
\subsubsection{DisableStoutBalancing}\label{disablestoutbalancing}}

If present, switches off the balancing.

\hypertarget{disablertr}{%
\subsubsection{DisableRTR}\label{disablertr}}

Switches off the three-step translation of \emph{T}-matrices, where a
general translation is decomposed into a rotation, \emph{z}-axial
translation, and then the inverse rotation. Instead, a one-step
transformation is performed by pre- or post-multiplying by a single
matrix containing the general translation-addition coefficients.

\end{document}
